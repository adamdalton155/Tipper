\chapter{Introduction}
\section{Investigating our idea}
The title of our project is Tipper, a mobile application written in React-Native JavaScript. With this project, we wanted to meet the goal of successfully transferring a sum of money from one user to another. Tipper was developed with waiters and waitresses in mind, with them being able to receive a payment directly into their own bank account as opposed to the tip being given to the business itself or split between the staff. When we were in the initial stages of developing the idea for the app, we spoke to employees of the hospitality industry, more specifically hotel bar and wait staff. We asked them how do they feel about the tipping culture in Ireland both on the side of receiving the tip and also how the business handles the division of tips.

We asked questions such as 'How do you get tips from customers?', 'Are tips received often?' and 'How do you split up the tips at the end of your work shift?' The answers we received painted a very clear picture about how the staff felt. They believed that the tipping culture was important to have as it does help to combat the increasing cost of living crisis that is currently happening. They also noted an increase in the frequency of amount of times that they received some sort of bonus for their service but that also often depended on where the customer was from. For example, if a customer came into their business and originated from the United States, the member of staff felt that they were more likely to receive a tip where as if the customer was Irish, they understood that they were less likely to receive any bonus. Ultimately, it seems this just a cultural difference. As of 2023, in the United States the federal minimum wage is seven dollars and twenty-five cents per hour or two dollars and thirteen cents per hour if the federal wage plus the tips is equal to or more than minimum wage\cite{MinWageUS}. 

In Ireland, as of 2023, the minimum wage is eleven euros and thirty cents\cite{MinWageIreland}. If we were to convert the seven dollars and twenty-five cents into euros, it equates to around six euros and sixty cents at the time of calculation. The difference between an employee in the US and Ireland being around four euros and seventy cents per hour. From this, we can see the cultural differences playing role in whether a tip is given to a member of staff. A really interesting detail that we learned about how tips are divided is that it's usually divided up between the members of staff either at the end of the shift or at the end of the business day, depending on when the shift ends. Furthermore we learned that a lot of staff feel this was doesn't work very well. One member of staff on a team of five may earn forty euros in tips where as the other four people may bring in ten euros and then that is split between everyone resulting in feelings that they are losing out on money that they earned, or that they are working harder to receive the tip than other members of staff and they should take home specifically what they earned. Because of this, it became clear that the app would need to pay users directly instead of potentially going into someone else's pocket in an attempt to try and protect the worker.

A contributing factor that we didn't expect to learn was the way payments are done. Nowadays the majority of transactions are being made electronically with either debit/credit cards, virtual cards stored on the phone through Google/Apple Pay or through a third party service such as Revolut. What we have also learned is that many businesses are no longer or announced that soon they will not accept any form of cash payment such as Starbucks. This is something that world governments are trying to push for, a cashless society. When tipping is a large part of say the US's economy, there is a huge issue that would emerge when they do go fully into a cashless society.

From speaking to employees in the hospitality sector, we gained an understanding of how tips are made, the culture behind tipping and, how tips are divided and mainly how they felt about tipping in Ireland. By gathering these requirements, we were then able to start to understand exactly would be required from an app like this. We understand that first, we need to be able to ensure that the user can securely and efficiently transfer money from one user to another. From this understanding, we must look for ways of connecting two users. Some applications use mobile numbers such a Revolut where as other applications use Quick Response (QR) codes. When looking at these options, GDPR and security becomes a major concern. Handing out personal identifiable information to multiple people per day has the possibility of becoming a security issue. Then we looked at usernames but that retracts from the ability to quickly to transfer the money which also raises other questions such as what if they input the wrong username or having to ask the receiver to spell out the username which takes them away from doing their job. We felt that if a business sees this app as potentially distracting it's employees over smaller details such as trying to spell out the username, the adoption of this platform may be hindered. 

After understanding how we will connect two users, we then needed to figure out what is exactly required of the user. We knew that security of the users information and their money was incredibly important so we had to decide how do we make this secure. So, we looked towards other applications and security systems. Optional Authentication, Two-Factor Authentication and Simple Security have been the standards of security at different times for different reasons and purposes, so we had to evaluate these options and decide which would meet the requirements. This also ties into the other information required from the user. Of course we need some information such as a name, an e-mail address and a password but then what else would we require from a user. First of all, we felt that it was important to make sure that we didn't take too much information from the user and only ask for what we need. It was also important to make sure that we would take into account future iterations of this application. If we wanted to continue to develop this application past college, we needed to make sure we took scalability into account by gathering the right amount of information that would enable to us to develop features later down the line. For example, what if we needed to implement another layer of security to a users account, what would we need to do that?

When we fully understood what would be required from a user of Tipper, we asked ourselves how do we separate a user who only needs to make payments from someone who would be receiving the payments. We knew that how they use the app would be different and so what they see when they use app must also be different. So now we knew that there are two different types of users of the application. From this we quickly understood that we would have to develop features that is unique to the user depending on what type of user they are. An example of this is someone who is a waiter, needs a way to receive the payment where as a customer of the business needs a way to send the payment. We began to look at platform dependent and platform independent ways of solving this issue. If the users was on an iOS device, do they use Apple Pay, on Android, do they use Google Pay. As for a platform independent solution, Stripe could be implemented. We needed to evaluate the potential undertaking of implementing one of these payment solutions while also taking into account the security, what platform is the user running the application on and the features of the service to identify what suits our requirements the best.

Next, we needed to find out what is the best way to store users information on a database. We looked at what we needed from a database, what exactly we needed it to be able to do and finally compare the features of different databases to understand, like the payment solution, what suits our application the best. We looked at various options such as MongoDB, Firebase, Microsoft SQL Server and AWS Amplify. We noted that each had their pros and cons. As an example of this, one database had much more detailed documentation and resources online where as others had built in functions that handled login, verification and updating of login information. This is where understanding the requirements became very important. We had to weigh up the pros and cons of each database, see what they offered in terms of features and then pick what we felt would be the most suitable.

We had to look at how they would calculate the tip. We noted that iOS and Android devices had the ability to scan text using a camera. But also noted that the text on a receipt was a lot and finding a way to have the bill total calculated by picking out that information might cause issues. We also needed to look at a way for calculating tips. We understand that some countries tip in increments of five or two point five percent. From speaking to the employees in these businesses, they noted that percentage calculations based on a bills total was not really the norm but rather people generally just left a euro note. Understanding the user base and how they go about using the app was something we didn't initially think of but was an aspect we did have to start considering.

How we were going to build the application was very important. We understood that from the start that it was going to be a mobile application that needed to run on mobile devices (iPhone/Android). Prior to starting development, we weren't very familiar with the frameworks used for mobile application development. We knew that we would be writing the app in JavaScript. Two frameworks stood out and they were Ionic and React-Native. From other work done we were more familiar with ReactJS which is just React-Native but for developing the front-end on a website. With the goal of developing a responsive application we had to look at how responsive each framework was, the documentation/resources for both and finally what served our purpose.

When we conducted our research, investigated the technology and discussed different features it became clear what the application needed to do, who it was for and how we should try and go about doing it. To sum up what Tipper needs to do is transfer a sum of money from a users account to the receivers account in a easy to use fashion that allows the users to connect to each other securely while offering the systems to allow this connection to happen. Users on both sides of the app need to be able to create an account based on the type of user that they are and have security measures in place such as Two-Factor Authentication or something similar. We feel if we can achieve these goals and have a working application with these features, we hope that this application would be considered a well designed and feature complete application.